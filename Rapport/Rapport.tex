\documentclass[final]{rapport1}
\usepackage[utf8]{inputenc}
\usepackage[export]{adjustbox} 
\usepackage{pdfpages}
\usepackage{graphicx}
\usepackage{authblk}
\usepackage[danish]{babel}
\usepackage{csquotes}
\usepackage[nottoc]{tocbibind}
\addto{\captionsdanish}{\renewcommand{\abstractname}{Abstract}}


\begin{document}
\begin{titlepage}
%\includepdf{forside.pdf}
\end{titlepage}


\includepdf[width=400pt]{forside.pdf}



 
\baselineskip= 15pt

\begin{abstract}

In this report we address our workgroups attempt at designing a digital solution for broadening the audience for both the digital and physical offerings, that Statens Museum for Kunst(SMK) provides. We use a multitude of idea generation techniques, to arrive at a product, and then interviews with SMK. We propose an enhanced web-presence, for image viewing and discussion, based on feedback from SMK and visits to the museum. The report concludes with the proposal of a system in which users can comment and interact with the different artworks on the website.


\end{abstract}
\clearpage
\tableofcontents
\chapter{Forord}


\chapter{Problemformulering}

\section{Problemfelt}

\section{Problemstilling}


\section{Problemformulering}
\begin{itemize}
\item 

\item 

\item 
\end{itemize}

\chapter{Målgruppe}

%\begin{picture}

%\includegraphics[width=300pt]{minverva.png}

%\end{picture}
%\begin{center}
%\tiny Figure 1. Minerva modellen
%\linebreak
%\linebreak
%\linebreak
%\end{center}


\section{Målgruppens efterspørgsler}


\clearpage
\chapter{Problemanalyse}

\chapter{Løsning}
\section{Teknisk Analyse}
\subsection{Kommentarsystem}


\subsection{Crowd-sourcet}

\subsection{Single Page Application}

\subsection{Backend}

\subsection{Programmeringssprog}
\begin{itemize}
\item 
\item 
\item 
\item 
\item 
\end{itemize}

\subsection{White- og Blackbox testing}

\subsection{Application Programming Interface(API)}
\clearpage
\section{Design}

\subsection{Index side}

\subsection{Værkforum}



\subsection{Zoomfuktion}


\section{Teknisk beskrivelse}
\subsection{Flowchart}

\subsection{Data}
\begin{itemize}
\item 
\item
\item 
\begin{itemize}
\item 
\item 
\end{itemize}
\end{itemize}

\subsection{Arkitektur}
\begin{itemize}
\item 
\item 
\end{itemize}

\subsection{Brugergrænseflade}
\subsubsection{Lyttere}

\subsubsection{Komponenter}


\subsection{Begrænsninger}

\chapter{Afprøvning}

\begin{enumerate}
\item 
\item 
\item 
\item

\end{enumerate}

\chapter{Brugervejledning}
\section{Eksempel}

\section{Brugsscenarie}

\chapter{Konklusion}

\chapter{Litteratur}

\end{document}






